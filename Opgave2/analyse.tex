\section{Analyse af modellerings og simulerings processer}

\subsection{K. Erleben - Simulering og modellering af robotter og mennesker}
Aktiklen har en kort gennemgang af problemet i den virkelige verden. Derefter
gennemgå den de den matematiske model for problemet ved hjælp af en
idealiserings proces. Herefter beskrives den næste proces, diskretisering, hvor
ved der kommer frem mod den diskrete model. Til sidst sluttets der af med at
fortælle lidt mere om hvordan virkelighedden ser ud.

Der er ingen simulering og derfor ingen resultater, hvilket også medføre at
der ikke kan laves validering. Aktiklen kommer kon med en generel udgave af
algoritmen og ikke en implementeret i et program, hvilket er skyld i at der
ikke kan laves simulering og verificering.


Ved at implemetere algoritmen i et program ville vi kunne lave en simulering og
derved kunne få nogle resultater som kunne validere på om vi med programmet
kunne styre en skelet til at tage en kaffekop.

\subsection{Hsu og Keyser - Piles of Objects}
\paragraph{(a)}
Artiklen beskriver en mere effektiv metode til at simulere ensformige der ligger
i ``bunker''. Dog er formålet ikke at opnå den samme konfiguration af objekterne
som den tilsvarende startkonfiguration vil føre til -- derimod blot at opnå en
effekt som ser realistisk ud.

Modellen der bruges er en standard fysiksimulation med friktion, tyngdekraft og
lignende. Det videnskabelige arbejde går på, at have lavet en ændring i
diskretiseringen.

I denne diskretisering kan objekterne deaktiveres, hvorved de
ikke vil bevæge sig før de bliver vækket igen senere -- enten fordi at bunken
begynder at skride, eller fordi bunken bliver ramt af et objekt med høj fart.
Denne deaktivering gør, at disse objekter (eventuelt midlertidigt) ikke behøves
simmuleret. På denne måde opnår man en performanceforbedring, som eventuelt kan
bruges til at bruge mere realistiske friktionssimmuleringer.

De verificerer at modellen er acceptabel, ved at sammenligne deres optimerede
diskretisering med en mere standardiseret diskretisering. Valideringen består
hovedsageligt i eyeballing.

\paragraph{(b)}
Det videnskabelige arbejde i artiklen består i at eftervise hypotesen ``en
simulation hvori vores optimering udnyttes vil være omtrent lige så præcis som
en fuld simulation der ikke gør''.

Vi mener egentlig ikke der er noget umiddelbart kritisabelt i artiklen -- men en
stor del af artiklen indeholder i højere grad en beskrivelse af deres optimering
end noget egentlig videnskabeligt arbejde.
