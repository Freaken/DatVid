\section{Computerhukommelsens betydning}
Turing understreger flere gange at størrelsen på computerens hukommelse betyger meget for dens performance i Turing testen. Hvorfor mener Turing at dette er vigtigt, og hvor meget hukemmelse mener Turing vil kunne give en god performance i Turing testen?

Turing mener at inden for de næste 50 år, burde $10^9$ bit være tilstrækkeligt til at implementere en optimeret version af hjenen, i forhold til at gennemføre turing-testen. Han mener at med $10^9$ bit ville den kunne klare testen med 30\% succesrate.
Turing valgte tallet $10^9$ i og med han mente at menneskets hjerne for at klare testen skulle bruge ca $10^{10} - 10^{15}$ bit. 
Tallet er dog blevet revideret siden hen da vi eller ville have opnået dette tal for længe siden.
Turing skriver og jeg citere \footnote{Modellering og simulering af den menneskelige hjerne, Turing 1950}``The Critism that the machine cannot have much deversity of behaviot is just a way of saying that it cannnot have much storage capacity'' Eller simplificeret siger han at jo mere hukommelse vi tildeler maskinen jo mere avancerede imitations-programmer kan vi konstruere.
