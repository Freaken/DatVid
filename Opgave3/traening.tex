\section{Træning af computere}

En af måderne at konstruere en computer, der kan bestå Turing testen, er at
benytte læring. Hovedpointen er at lade computeren simulere den måde, som den
menneskelige hjerne lærer på.

I Kringelbachs artikel drøftes forskellen mellem den menneskelige hjerne og
de kunstige neurale netværk, som kan simulere denne.

Hvor langt mener Kringelbach at man i 2010 er fra at kunne modellere og
simulere den menneskelige hjerne – og hvorfor?


Vi er noget et godt stykke med udviklingen at kunstige hjerner, men vi er
ikke helt fremme endnu hvor vi succesfuld kan simulere en fuld hjerne.

Der har været forsøg med at lave meget store nuerale netværk der fungere med
binære neuroner og der har været forsøg med enkle højt detaleret neuroner.
Begge dele kraver meget stor regnekraft og det har ind til videre er der ingen
der har funden en gylden mellemvej hvor net netværket har været stort nok
og at de enkle neuroner detaljeret nok til at kunne simulere en hjerne i fuld
evne.

Der finde dog fungerende systemer som bruger neurale netværk. Google
bruger til deres oversættelse service et system der ud fra nogle nogle
oversatte dokumenter selv kan regne ud hvordan man skal en tekst skal
oversættet. Når Google Oversæt genererer en oversættelse, søger den
efter mønstre i flere hundrede millioner dokumenter for at finde frem
til den bedste oversættelse. Ved at registrere mønstre i dokumenter,
der allerede er blevet oversat af rigtige oversættere, kan Google
Oversæt foretage et intelligent gæt på, hvad der vil være en passende
oversættelse. Denne proces med at søge efter mønstre i store mængder
tekst kaldes "statistisk maskinoversættelse". Eftersom oversættelserne
genereres af maskiner, vil ikke alle oversættelser være perfekte. Jo flere
oversættelser, der er oversat af rigtige oversættere, som Google Oversæt
kan analysere på et bestemt sprog, jo bedre bliver oversættelseskvaliteten.
Det er grunden til, at oversættelseskvaliteten kan variere mellem
sprogene.\footnote{http://translate.google.com/about/intl/da\_ALL/}

