\section{Jeres prioritering af indvendingerne mod Turing testen}
Vi har startet med at dele alle argumenterne ind i tre meget generelle
kategorier, som klart størstedelen af alle argumenter ``mod'' Turing testen vil
falde ind under (dog muligvis under mere end én kategori).

\begin{itemize}
\item Argumenter, som ikke mener at det er muligt, at konstruere en computer der
kan bestå Turing testen.
\item Argumenter, som ikke mener at det beviser noget, selv hvis det skulle
lykkedes en computer at bestå Turing testen.
\item Argumenter, som ikke mener at man bør forsøge at konstruere computere, som
kan bestå Turing testen.
\end{itemize}

De argumenter vi har fået, som falder i den første kategori har den (oftest
implicitte) form ``Vi antager $X$. Herudfra kan vi udlede at computere ikke
kan bestå Turing testen''. Eksempler på antagelsen kan her være ``mennesker
kan beregne enhver funktion'', ``kun mennesker kan have følelser eller
humor'', ``mennesker har oversanselige reception'' eller ``mangel-på-regler
indvendingen''. Alle antagelserne er desuden enten meget svære eller umulige
at falsificere på nuværende tidspunkt. Spørgsmålene er naturligvis meget
interessante fra et filosofisk synspunkt, men fra et rent naturvidenskabeligt
synspunkt kan de derfor ikke bruges til noget.

Fra den anden kategori er eksempelvis ``The Chinese Room''-argumentet,
Blockhead-argumentet samt Lady Lovelace's indvending. Her er det centrale
omdrejningspunkt hvordan intelligens og bevidsthed defineres. I en
naturvidenskabelige kontekst ville det være nærtliggende at lave definitionen på
en sådan måde, at rent faktisk ville være muligt at skelne intelligente systemer
fra meget komplekse systemer. For enhver definition som opfylder dette, vil en
enhed, som kan klare Turing testen er netop også være intelligent - argumenterne
bliver dermed nød til at afhænge af en (fra et naturvidenskabeligt synspunkt)
ikke særligt brugbar definition.

Fra den sidste kategori er der kun et enkelt eksempel, nemlig ``hovedet i
sandet''-argumentet -- man kan dog sagtens forestille sig mange variationer af
argumentet. Eksempelvis kunne man argumentere for, at eksistensen af
intelligente computere vil medføre en devaluering af samfundets opfattelse af
det enkelte menneskes værdi, eller man kunne argumentere ud fra et moralsk
regelsæt så som i Dune-universet af Frank Herbert, hvori der indgår dogmet
``Thou shalt not make a machine in the likeness of a human mind''.

Vi er i vores gruppe en meget pragmatisk tilgang til kunstig intelligens - vi
kan ikke se nogen grund til at der skulle være en universel grund til at det
ikke skulle kunne lade sig gøre at kopiere menneskelig intelligens. Vi
anerkender at spørgsmålene er meget interessante fra et filosofisk synspunkt,
men finder dem ikke relevante fra et naturvidenskabeligt synspunkt. Vi synes
derfor at ``hovedet i sandet''-argumentet er det vigtigste -- og hvis presset,
så er ``The Chinese Room''-argumentet og ``Indvending om nervesystemets
kontinuitet'' de næstvigtigste.
