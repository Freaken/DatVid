\section{Reflektion}
Hvad ligger bag udtrykket Computer Science, er dette felt overhovedet videnskab?
Dette har været et åbent spørgsmål siden feltets fødsel, og det har desuden
affødt flere afgreninger på forskellige læreanstalter igennem tiden. Både Stewart og
Denning arbejder med dette spørgsmål i de artikler vi har fået udleveret\footnote{Stewart (1994), Denning (2005), Denning (2007)}.

Denning ser forbi computeren som et værktøj og håndtere Computer Science som værende videnskaben omkring processer og
information. På denne måde finder han frem til at Computer Science ikke er bundet til computeren i sig selv, men at
feltet griber ud og har nære forbindelser til de andre naturvidenskabelige felter. Derfor er det efter Dennings opfattelse af
Computer Science meget nært at udføre simulationer af modeller af processer og information fra andre felter såsom biologi
eller fysik.

Stewart, i 1994, mener ikke at Computer Science er rigtig videnskab, og fremsætter Poppers Kriterion som grundlag for sin
argumentation. Han mener dog, at dette blandt andet er et spørgsmål om hans samtids kultur omkring feltet og at
Computer Science godt kan blive en videnskab. En af de ting han fokusere på er simulationer af modeller fra andre
felter. Han siger blandt andet at ``... An algorithm is proposed, it is applied to real or contrived data, and the results are
observed to see whether the algorithm seems to imitate a biological organism in some ill-defined way.''
Det er tydeligt ud fra tonen i hans tekst, at han ikke mener at den gældende praksis i hans samtid opfylder kriterierne for
at være videnskab, og han synes desuden ikke at have den store tiltro til simulationer som værende rigtig videnskab. Derved
er han i kontrast til hvad vi har læst fra Denning.

Denne kontrast kommer sandsynligvis af en forskel i tid og at Computer Science (stadigt) er et forholdsvist ungt felt.
Man kan umiddelbart sige at Stewart og Denning er uenige om hvorvidt simulationer som disse passer ind i feltet, men
denne uenige udspringer højst sandsynligt af en forskel i hvad deres opfattelse af Computer Science ud fra deres samtid.
