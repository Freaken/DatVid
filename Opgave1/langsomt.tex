\subsection{For langsomt til målinger/observationer}
I astronomien og astrofysikkens verden foregår ting henover skalaer der er
langt større end hvad vi kan arbejde med her på jorden. En af disse skalaer
er tid; for eksempel den tid det tager for en galakse at formes, en stjerne
at dø ud eller blot det at et lysglimt sendes afsted, reflekteres og kommer
tilbage.
Hvis man vil efterprøve sine fysiske modeller for hvordan vores univers
opfører sig, kan det derfor være nødvendigt at være i stand til at spole
frem og tilbage i eksperimentet. Da fysikken stadigt ikke har fremskabt en
tidsmaskine, er en computer-baseret simulation derfor den eneste mulighed i
mange tilfælde.
