\subsection{Planet-simulator}
Vi har skrevet planetsimulatoren i haskell, og den kan køres med eksempelvis
{\tt runghc Newton.hs}. Simulatoren kan simulere et vilkårligt antal dimensioner
og understøtter at planeterne kan have forskellig masse - dog simplificerer
simulationen planeterne til punktmasser. Beregningen foregår ud fra følgende
formler:

\begin{gather*}
\begin{split}
\mathbf{a}_{p,t_0} &= \frac{\mathbf{F}_{p,t_0}}{m_p}\\
&= \frac{\sum_k^{P \backslash \{p\}} \frac{G*m_k*m_p}{|\mathbf{r}_{k,t_0} - \mathbf{r}_{p,t_0}|^2} *
\frac{\mathbf{r}_{k,t_0} - \mathbf{r}_{p,t_0}}{|\mathbf{r}_{k,t_0} - \mathbf{r}_{p,t_0}|}}{m_p} \\
&= \sum_k^{P \backslash \{p\}} \frac{G*m_k}{|\mathbf{r}_{k,t_0} - \mathbf{r}_{p,t_0}|^3}
* (\mathbf{r}_{k,t_0} - \mathbf{r}_{p,t_0})
\end{split} \\
\\
\mathbf{v}_{p,t_0+s} = \mathbf{v}_{p,t_0} + \mathbf{a}_{p,t_0}*s \\
\mathbf{r}_{p,t_0+s} = \mathbf{r}_{p,t_0} + \mathbf{v}_{p,t_0}*s
\end{gather*}

Her er $\mathbf{r}_{p,t_0}$, $\mathbf{v}_{p,t_0}$, $\mathbf{a}_{p,t_0}$,
$\mathbf{F}_{p,t_0}$ og $m_p$ henholsvis stedvektoren, hastighedsvektoren,
accelerationsvektoren, den resulterende kraft og massen tilknyttet planeten $p$
til tiden $t_0$. $P$ er mængden af planeter og $s$ er længden af tidstrinnene
og $G$ er gravitationskonstanten.

\definecolor{grey}{RGB}{75,75,75}
\lstset{
    language=haskell,                 % choose the language of the code
    basicstyle=\ttfamily\footnotesize % the size of the fonts that are used for the
    keywordstyle=,
    numbers=left,                     % where to put the line-numbers
    numberstyle=\footnotesize,        % the size of the fonts that are used for the
    numbersep=5pt,                    % how far the line-numbers are from the code
    backgroundcolor=\color{white},    % choose the background color. You must add
    showspaces=false,                 % show spaces adding particular underscores
    showstringspaces=false,           % underline spaces within strings
    showtabs=false,                   % show tabs within strings adding particular
    frame=single,                     % adds a frame around the code
    breaklines=true,                  % sets automatic line breaking
    breakatwhitespace=false           % sets if automatic breaks should only happen at
}
\lstinputlisting{Newton.hs}
