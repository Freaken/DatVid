\subsection{Uetiske/uønskede}
Der kan være mange grunde til at et eksperiment kan være uetisk eller på anden
vis uønsket at udføre.

\subsubsection{Eks. 1: Corrupted Blood incident}
Dette ``eksperiment'' begyndte den 13 september 2005, da Blizzard intruducere et nyt
spilområde\footnote{http://en.wikipedia.org/wiki/Corrupted\_Blood\_incident}.
I enden af dette spilområde var det muligt at blive ``smittet'' med ``Corrupted
Blood''. ``Corrupted Blood'' gjorde at spilleren fik skade over tid, og den
spredte sig til andre spillere i nærheden af spilleren. Som udgangspunkt blev
spillerne automatisk kureret når de forlod området, men det lykkedes altså for folk
at komme ud af spilområdet uden at blive kureret. Dette resulterede i en hurtig spredning
af ``sygdommen'' i spillet, hvilket inden for kort tid nåede epidemisk status i spilleverdenen.
Programmørerne blev nød til at indsætte karantæne på specielt befolkede områder (storbyer mm.)
og måtte ``rense ud'' før disse områder kunne genåbnes.

Efterfølgene blev fænomenet taget op af mange epidemi forskere,
\footnote{http://www.highbeam.com/doc/1G1-171623497.html} som en god simulering
af hvordan en epidemi ville kunne udvikle sig. Således blev der altså (mod spiludviklernes
gode vilje) udført et eksperiment af propotioner der ville have været både uetiske og
uønskede at udføre i den virkelige verden.

\subsubsection{Eks. 2: Minefelts simulation}
Et mere kontrolleret eksempel på et forsøg der ikke ville være ønsket at udføre i
virkeligheden er en simulation af en gruppe af infanterister der skal forcere et
minefelt. I denne sammenhæng er det uønsket at sende en gruppe infanterister over
et minefelt for at samle datapunkter, og det er derfor mere ønskeligt at opstille
en model over situationen for derefter at køre den på en computer.
