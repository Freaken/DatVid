\section{Klima}

\subsection{De bedste klimasimuleringer i dag}
Vi har valgt at lave et optimistisk ``best case'' scenario. Vi antager
derfor at vi får 3 måneders kørselstid på verdens bedste supercomputer, som
kan køre med $2.57 petaflop/s$.\footnote{Dette er den maksimalt opnåede ydelse
ifølge http://www.top500.org/lists/2010/11. For et realistisk estimat skal nok
bruges et noget mindre tal.}

Vi antager at den bedste klimamodel er baseret på diskrete tidsenheder og en
inddeling af jordoverfladen i diskrete områder. Vi antager derudover at
det tager omtrent 100 kiloflop at beregne et område per tidsenhed. Dette er i
høj grad et subjektivt estimat og er baseret på vores umiddelbare opfattelse af
kompleksiteten af en brugbar model samt antallet af parametre i en fornuftig algoritmer.

Herudfra kan vi udregne at der kan udregnes i alt $2.57 petaflop/s * 3 måneder /
100 kiloflop \approx 2.03*10^14$ områder i en enkelt tidsenhed. Hvis vi gerne
ville simulere 100 år med områder på omkring $1 km^2$ kan man altså simulere det
med tidstrin på omkring 2.2 timer.
