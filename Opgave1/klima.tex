\section{Klima}
\subsection{Fordelene ved klimasimuleringer}
Et af de store politiske spørgsmål i dag hvilken grad mennesket er skyld i
klimaforandringerne og hvad vi bør gøre ved dette. I denne sammenhæng er
den mest åbenlyse fordel ved klimasimuleringer naturligvis, at de gør det
muligt at få en idé om hvordan menneskets handlinger har indflydelse på vores
planets fremtidige tilstand. Vi kan bruge dem til at prøve at opnå det bedste
kompromis mellem at beholde vores nuværende adfærd og beholde vores nuværende
klima.

\subsection{Kan den udleverede simulering bruges til noget?}
{\bf Meget} lidt. Den er alt for simpel til at give et realistisk billede af
noget som helst praktisk. Den kan dog bruges til at vise, at et større indtil af
$CO_2$ i atmosfæren medfører højere temperatur.

\subsection{5 forbedringer}
\begin{itemize}
\item Skyer
\item En kugleoverflade på størrelse med jorden i stedet en 1d overflade.
\item Lokale geografiske forskelle svarende til jordens
\item Golfstrømmen
\item Jordens hældning/årstider
\end{itemize}

\subsection{De bedste klimasimuleringer i dag}
Vi har valgt at lave et optimistisk ``best case'' scenario. Vi antager
derfor at vi får 3 måneders kørselstid på verdens bedste supercomputer, som
kan køre med $2.57 petaflop/s$.\footnote{Dette er den maksimalt opnåede ydelse
ifølge http://www.top500.org/lists/2010/11. For et realistisk estimat skal nok
bruges et noget mindre tal.}

Vi antager at den bedste klimamodel er baseret på diskrete tidsenheder og en
inddeling af jordoverfladen i diskrete områder. Vi antager derudover at
det tager omtrent 100 kiloflop at beregne et område per tidsenhed. Dette er i
høj grad et subjektivt estimat og er baseret på vores umiddelbare opfattelse af
kompleksiteten af en brugbar model samt antallet af parametre i en fornuftig algoritmer.

Herudfra kan vi udregne at der kan udregnes i alt $2.57 petaflop/s * 3 måneder /
100 kiloflop \approx 2.03*10^14$ områder i en enkelt tidsenhed. Hvis vi gerne
ville simulere 100 år med områder på omkring $1 km^2$ kan man altså simulere det
med tidstrin på omkring 2.2 timer.
