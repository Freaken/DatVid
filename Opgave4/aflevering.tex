\documentclass[11pt,fleqn]{article}
\usepackage[a4paper, hmargin={2.8cm, 2.8cm}, vmargin={2.8cm, 2.8cm}]{geometry}  % Geometri-pakke: Styrer bl.a. maginer                              %
\usepackage[utf8]{inputenc}                                         % Lidt kodning så der ikke kommer problemer ved visse konverteringer            %
\usepackage[babel, lille, nat, da, farve]{ku-forside}               % KU-forside med logoer                                                         %
\usepackage{listingsutf8}
\usepackage{amsmath}
\def\HyperLinks{                                                    %  Hyperlinks-pakke, der laver referencer til links og tillader links til www   %
\usepackage[pdftitle={\TITEL},pdfauthor={\FORFATTER},               %  Der er foretaget et lille trick så pakken indlæses efter                     %
pdfsubject={\UNDERTITEL}, linkbordercolor={0.8 0.8 0.8}]{hyperref}} %  titlen defineres.                                                            %

\usepackage{lastpage}
\usepackage{fancyhdr}
\usepackage[toc]{appendix}
\usepackage{pdfpages}

\forfatter{Davy Eskildsen, Emil Johanson, Thorbjørn S. Kaiser, Mathias Svensson}
\dato{25 November 2010}
\titel{Gruppeopgave 1, Oplæg 1: Computerbaserede simuleringer}
\undertitel{}

\HyperLinks % Henter hyperlinks-pakke og sætter pdf-titel mm. til at svare til de just definerede

\pagestyle{fancy}
\lhead{{\small \FORFATTER}}
\chead{}
\rhead{}
\cfoot{\footnotesize Side \thepage \ af \pageref{LastPage}}

\setcounter{secnumdepth}{1}
\setcounter{tocdepth}{2}
\mathcode`\*"8000{\catcode`\*\active\gdef*{\cdot}}


\begin{document}

\maketitle

\tableofcontents

\thispagestyle{empty}

\paragraph{1.} Som udgangspunkt må frisørsalonen ikke registrere CPR-numrene,
medmindre kunderne har givet tilladelse til det\footnote{Persondataloven §3, nr.
8}. En simpel løsning på dette ville være en afkrydsningsboks med ``Jeg
accepterer vilkårene for siden'', og et link til disse vilkår.

\paragraph{2.} Som i de fleste sager i forbindelse med persondata, så
er det tilladt, hvis den registrerede har givet sit samtykke. Her er der
tale om et portrætbillede, der som udgangspunkt altid kræver samtykke.
Den eneste ting som taler i den modsatte retning er at der er tale om en
offentlig person. Vi har ikke kunne finde noget konkret i dansk lovgivning
som omhandler offentlige personer, men vi antager at de samme rettigheder i
høj grad gælder. Dette ses endnu tydeligere, idet billedet til dels skal
bruges til markedsføringen. Hvis man sætter det en smule på spidsen, så
kan det reelt sammenlignes med paparazzi-fotograferingen, hvilket er imod
menneskerettighederne\footnote{\url{
    http://www.pressefotografforbundet.dk/news/fotografi/934/paparazzibilleder-forbudt
}}.

\paragraph{3.} Som udgangspunkt er det altid ulovligt at registrere kunder uden
deres samtykke\footnote{Persondataloven §28} -- og endnu kraftigere så skal den
registreredes orienteres om at de er blevet registreret.

\paragraph{4.} Som udgangspunkt kræves igen accept. Det er dog muligt, at
undersøgelsen falder ind under emnet ``forskning og statstik''. Hvis det er
tilfældet må de dog gerne videregive anonymiserede data, såfremt Datatilsynet
har givet tilladelse hertil.

\paragraph{5.} Grethe har ret til at blive orienteret og få indsigt i
undsøgelsen. Hun har ret til at få rettet forkerte oplysninger og har desuden
altid ret til at få oplysningerne slettet. Hun kan desuden klage til
Datatilsynet.

\paragraph{6.}

\paragraph{7.}

\end{document}
