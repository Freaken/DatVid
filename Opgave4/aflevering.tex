\include{preamble}

\begin{document}

\maketitle

\tableofcontents

\thispagestyle{empty}

\paragraph{1.} Som udgangspunkt må frisørsalonen ikke registrere CPR-numrene,
medmindre kunderne har givet tilladelse til det\footnote{Persondataloven §3, nr.
8}. En simpel løsning på dette ville være en afkrydsningsboks med ``Jeg
accepterer vilkårene for siden'', og et link til disse vilkår.

\paragraph{2.} Som i de fleste sager i forbindelse med persondata, så
er det tilladt, hvis den registrerede har givet sit samtykke. Her er der
tale om et portrætbillede, der som udgangspunkt altid kræver samtykke.
Den eneste ting som taler i den modsatte retning er at der er tale om en
offentlig person. Vi har ikke kunne finde noget konkret i dansk lovgivning
som omhandler offentlige personer, men vi antager at de samme rettigheder i
høj grad gælder. Dette ses endnu tydeligere, idet billedet til dels skal
bruges til markedsføringen. Hvis man sætter det en smule på spidsen, så
kan det reelt sammenlignes med paparazzi-fotograferingen, hvilket er imod
menneskerettighederne\footnote{\url{
    http://www.pressefotografforbundet.dk/news/fotografi/934/paparazzibilleder-forbudt
}}.

\paragraph{3.} Der må som udgangspunkt ikke behandles følsomme oplysninger,
såsom helbredsmæssige forhold, hvilket her er tilfældet.
For at få lov til at gøre dette skal kunden i dette tilfælde have givet
sit udtrykkelige samtykke og i den forbindelse være orienteret om hvad
dataen bliver brugt til. Det er her ikke nok at argumentere ud fra indirekte
samtykke såsom at de er kunde, da der er tale om følsomme oplysninger.
Herunder gælder den registeredes ret til at kunne gøre indsigelse mod
at dataen behandles, hvilket kunden nægtes i kraft af at de ikke
orienteres.

\paragraph{4.} Som udgangspunkt kræves igen accept. Det er dog muligt, at
undersøgelsen falder ind under emnet ``forskning og statistik''. Hvis det er
tilfældet må de dog gerne videregive anonymiserede data, såfremt Datatilsynet
har givet tilladelse hertil. Igen gælder desuden at kunden skal gøres opmærksom
på denne databehandling og desuden har ret til at gøre indsigelse imod denne
hvorfor den øjeblikkeligt skal ophøre, også selvom informationen er anonymiseret.

\paragraph{5.} Grethe har ret til at blive orienteret og få indsigt i
undsøgelsen, hvilket hun desuden skulle have haft fra starten.
Hun har ret til at få rettet forkerte oplysninger og har desuden
altid ret til at få oplysningerne slettet. I dette tilfælde bør hun
tage kontakt til datatilsynet, da der er tale om ulovlig håndtering
af følsomme oplysninger.

\paragraph{6.}
Såfrem Niels Jensen ikke efter udvikling af siden har noget at gøre med den
konkrete indsamling eller behandling af data kan han ikke umiddelbart gøres
ansvarlig for hvordan siden bruges. Det er Lis Hansen der er både dataansvarlig og
databehandler. Det er for eksempel ikke Niels Jensens ansvar om Lis Hansen
sørger for at orientere de registerede og indsamle den lovpligtige samtykke,
hvorvidt Lis Hansen sørger for tilstrækkelig sikkerhed omkring serveren, osv.

\paragraph{7.}
\begin{enumerate}
\item Som udgangspunkt skal kundens kommunikation til hjemmesiden ske henover en
sikret forbindelse, for eksempel https, og det skal foregå igennem en unik
bruger identificeret ved et brugernavn og et hemmeligt password.
\item Der skal foretages logning af kundernes handlinger på hjemmesiden, såsom
læsning og ændring af oplysninger, booking mm.. Denne logning skal selvfølgelig
være sikret og skal kun kunne tilgåes af en administrator.
\item Alle eventuelle administratorkonti skal være unikt knyttet til en person
og kun en person, således at ansvaret for eventuelle lovbrud kan placeres.
\item Der skal overvejes hvorvidt det er nødvendigt at gemme personnummeret,
eller om en kombination af navn og telefonnummer er nok til unikt at identificere
en kunde. Der skal ikke gemmes mere information end højest nødvendigt.
\item Selve serveren skal være fysisk sikret og adgang til denne skal være kontrolleret.
Dette behøver selvfølgelig ikke at være mere drastisk end at den skal stå i et aflåst
lokale og at direkte adgang skal foregå igennem personlig konti og logges på computeren.
Serveren skal desuden ikke bruges til andet såsom almindeligt computerbrug, for at undgå vira,
og skal desuden både være udstyret med firewall og antivirus som skal opdateres jævnligt.
\item Dataen på serveren skal selvfølgelig være krypteret, således at man for eksemple ikke
bare kan tage en kopi af harddisken og læse den fra en anden computer.
\end{enumerate}
\end{document}
